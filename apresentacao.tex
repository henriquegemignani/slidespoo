\documentclass[brazil]{beamer}
\usepackage{beamerthemesplit}
\usepackage[brazilian]{babel}
\usepackage[utf8]{inputenc}
\usepackage{hyperref}
\usepackage{color}
\usepackage{xcolor}
\usepackage{amssymb}
\usepackage{amsmath}
\usepackage{fancybox}
\usepackage{ulem}
\usepackage{listings}
\usetheme{JuanLesPins}

\title{Objective-C vs C++}
\author{Thiago de Gouveia Nunes\\ Henrique Gemignani Passos Lima}

\begin{document}

\frame{\titlepage}

%-------------------------------------
\section{0. História!}
%-------------------------------------

\frame{
  \begin{center}
    \Huge Um pouco de história...
  \end{center}
}

\frame{
  Objective-C e C++ são "forks" da linguagem C com a implementação de POO, criadas em 1983.
  \begin{columns}
    \begin{column}{.5\textwidth}
      \begin{center}
        \textbf{C++} 
      \end{center}
      Implementação com código mais estático de POO, com foco em desempenho.
    \end{column}
    \begin{column}{.5\textwidth}
      \begin{center}
        \textbf{Objective-C}
      \end{center}
      Muito próxima do smalltalk-80 tanto em sintaxe quanto em dinâmismo.
    \end{column}
  \end{columns}
}

%-------------------------------------
\section{1. Semelhanças}
%-------------------------------------

\frame{
  \begin{center}
    \Huge Semelhanças
  \end{center}
}

\frame{
  \begin{itemize}
    \item Extensões de C: código em C pode ser compilado como Objective-C ou C++
    \item Suporte a programação orientada a objetos.
    \item Declaração e código podem ser misturados.
    \item Separação da declaração dos headers em 2 arquivos.
    \item Atributos \textit{public}, \textit{private}, \textit{protected}.
    \item Referênciação de memória.
  \end{itemize}
}

%-------------------------------------
\section{2. Classes}
%-------------------------------------

%-------------------------------------
\subsection{Sintaxe}
%-------------------------------------

\begin{frame}[fragile]
  point2d.h
  \begin{columns}
    \begin{column}{.5\textwidth}
      \begin{center}
        C++
      \end{center}
      \lstset{language=C++,basicstyle=\tiny}
      \begin{lstlisting}
        class Point2D {
          public:
            // Getters
            float x();
            float y();

            // Metodos
            Point2D Add(Point2D& right);

          private:
            // Atributos.
            float x_;
            float y_;
        };
      \end{lstlisting}
    \end{column}
    \begin{column}{.5\textwidth}
      \begin{center}
        Objective-C
      \end{center}
      \lstset{language=C++,basicstyle=\tiny}
      \begin{lstlisting}
        @interface Point2D {
            // Atributos.
            float x;
            float y;
        }

        // Getters
        -(float) x;
        -(float) y;
        
        // Metodos
        -(Point2D*) add: (Point2D*)right;
        @end
      \end{lstlisting}
    \end{column}
  \end{columns}
\end{frame}

\begin{frame}[fragile]
  \begin{columns}
    \begin{column}{.5\textwidth}
      \begin{center}
        C++ \\
        point2d.cc
      \end{center}
      \lstset{language=C++,basicstyle=\tiny}
      \begin{lstlisting}
        float Point2D::x() { return x_; }
        float Point2D::y() { return y_; }

        Point2D Point2D::Add(Point2D right) {
            return ...
        }
      \end{lstlisting}
    \end{column}
    \begin{column}{.5\textwidth}
      \begin{center}
        Objective-C \\
        point2d.m
      \end{center}
      \lstset{language=C++,basicstyle=\tiny}
      \begin{lstlisting}
        @implementation Point2D
        -(float) x { return self->x; }
        -(float) y { return self->x; }

        -(Point2D*) add: (Point2D*) right {
            return ...;
        }

        @end
      \end{lstlisting}
    \end{column}
  \end{columns}
\end{frame}

\begin{frame}
  A principal diferença entre as Classes do Objective-C para as do C++ é a existência de uma classe
  raiz no Objective-C.
\end{frame}

%-------------------------------------
\subsection{Foward Declaration}
%-------------------------------------

\begin{frame}[fragile]
  No C++ podemos usar uma classe não completamente definida usando uma declaração avançada dela. \\
  \lstset{language=C++,basicstyle=\tiny}
  \begin{lstlisting}
    class Bar; //forward declaration 
    class Foo { 
      Bar* bar; 
      public:
        void userBar(void);
    };
  \end{lstlisting}
  No Objective-C o mesmo efeito é obtido usando a palavra reservada \textit{@class}.
  \lstset{language=C++,basicstyle=\tiny}
  \begin{lstlisting}
    @class Bar; //forward declaration 
    @interface Foo : NSObject { 
      Bar* bar; 
    }
    -(void) userBar;
  \end{lstlisting}
\end{frame}

%-------------------------------------
\subsection{Public, Private, Protected}
%-------------------------------------

\begin{frame}
  Nas duas linguagens os conceitos de \textit{public}, \textit{private} e \textit{proteced} existem. \\
  Em C++, tanto métodos quanto atributos podem receber esses modificadores. O padrão é private. \\
  Em Objective-C só atributos podem ser modificados, com padrão sendo protected. Todos métodos são públicos.
\end{frame}

\begin{frame}
  Uma maneira de implementar um métodos como "privados" é não declará-los na interface. \\
  Os métodos ainda poderam ser chamados, mas sua existência está escondida.
\end{frame}

%-------------------------------------
\subsection{Static}
%-------------------------------------

\begin{frame}
  Em C++ podemos declarar atributos ou métodos com o modificador \textit{static}. Assim eles passam a ser de classe. \\
  O Objective-C não tem suporte a atributos de classe, mas é possível atingir o mesmo resultado criando uma variável global 
  na implementação e utilizando seus métodos para modificar essa variável.
\end{frame}

\begin{frame}
  Para criar um método de classe em Objective-C é só acrescentar o símbolo '+' na frente da sua declaração, no lugar do '-'.
\end{frame}

%-------------------------------------
\subsection{Métodos}
%-------------------------------------

\begin{frame}[fragile]
  O código abaixo exemplifica a criação e chamada de um método:
  \lstset{language=C++,basicstyle=\tiny}
  \begin{lstlisting}
    //prototype
    void Array::insertObject(void *anObject, unsigned int atIndex)
    
    //use with a "shelf" instance of the Array class and a "book" object
    shelf.insertObject(book, 2);
  \end{lstlisting}
\end{frame}

\begin{frame}[fragile]
  O Objective-C tem a mesma base de mensagens do Smalltalk, com uma diferença na declaração da mensagem. \\
  \lstset{language=C++,basicstyle=\scriptsize}
  \begin{lstlisting}
    -(void) insertObject:(id)anObject:(unsigned int)index
    
    //use with a "shelf" instance of the Array class 
    //and a "book" object
    [shelf insertObject:book:2];
    =============================================================
    -(void) insertObject:(id)anObject atIndex:(unsigned int)index
    
    //use with a "shelf" instance of the Array class 
    //and a "book" object
    [shelf insertObject:book:2]; //Error !
    [shelf insertObject:book atIndex:2]; //OK
  \end{lstlisting}
\end{frame}

\begin{frame}
  Não é possivel chamar um método que não existe em uma classe no C++, o programa nem é compilado. \\
  Já no Objective-C é totalmente possível, e um objeto que não implementa uma mensagem devolve uma Exception específica.
\end{frame}

\begin{frame}
  Assim como o \textit{this} no C++, utilizado para referênciar o objeto atual, temos o \textit{self} no Objective-C. \\
  O \textit{self} não é uma palavra reservada, mas sim um argumento oculto de cada método! \\
  No Objective-C a palavra reservada \textit{super} é utilizada para passar mensagens para a classe pai. \\
  Como no C++ temos herança múltipla, utilizamos o nome da classe pai para ter o mesmo efeito.
\end{frame}

\begin{frame}
  No C++ a assinatura de um método é composta do seu nome e tipo dos parâmetros ( alêm da presença do modificador \textit{const} ). \\
  Dado isso o C++ consegue implementar o overload de métodos. \\
  O Objective-C também consegue, mas para tanto os métodos devem ter labels diferentes para seus atributos.
\end{frame}

\begin{frame}[fragile]
  Assim como em C, o C++ tem ponteiros para funções, que é expandido para métodos:
  \lstset{language=C++,basicstyle=\scriptsize}
  \begin{lstlisting}
    class Foo
    {
      public:
      int f(float x) {...}
    };
    Foo bar
    int (Foo::*p_f)(float) = &Foo::f; //Pointer to Foo::f
    (bar.*p_f)(1.2345); //calling bar.f(1.2345);
  \end{lstlisting}
\end{frame}

\begin{frame}
  No Objective-C um novo tipo é introduzido, o \textit{Selector}. \\
  Esse tipo é transformado numa \textit{string} utilizando uma tabela de métodos criada na compilação de uma classe. \\
  Com isso podemos verificar a existência dos métodos de uma classe.
\end{frame}

\begin{frame}
  É possível definir valores padrões para os parâmetros de métodos em C++.\\
  Isso não é possível em Objective-C. \\
\end{frame}

\begin{frame}
  Em C++ podemos declarar parâmetros anônimos no protótipo de um método, mas em Objective-C isso é impossível.
\end{frame}

\begin{frame}
  O Objective-C aceita mensagens para o objeto \textit{nil}. Essas mensagens são ignoradas. \\
  Em C++, utilizar qualquer objeto cujo ponteiro vale 0 leva a comportamento indefinido.
\end{frame}

%-------------------------------------
\section{3. Herança}
%-------------------------------------

\begin{frame}
  Como uma boa linguagem orientada a objetos, tanto o C++ quanto o Objective-C tem herança. \\
  A maior diferença é a herança múltipla do C++, que não está presente no Objective-C.
\end{frame}

\begin{frame}
  Todos os métodos do Objective-C são virtuais por padrão. \\
  No C++, o modificador \textit{virtual} deve ser usado para virtualizar um método. \\
  Um método virtual pode ser sobrescrito por uma classe filha.
\end{frame}

%-------------------------------------
\subsection{Interfaces}
%-------------------------------------

\begin{frame}[fragile]
  O C++ não tem interfaces, mas podemos chegar nesse resultado implementando uma classe abstrata que não defina nenhum método. \\
    \lstset{language=C++,basicstyle=\scriptsize}
  \begin{lstlisting}
    class MouseListener
  {
    public:
    virtual bool mousePressed(void) = 0; //pure virtual method
    virtual bool mouseClicked(void) = 0; //pure virtual method
  };
  
  class Foo : public MouseListener {...}
  \end{lstlisting}
\end{frame}

\begin{frame}[fragile]
  O Objective-C usa um artificio chamado de \textit{protocol} para implementar interfaces.
      \lstset{language=C++,basicstyle=\scriptsize}
  \begin{lstlisting}
  @protocol MouseListener
    -(BOOL) mousePressed;
    -(BOOL) mouseClicked;
  @end
  
  @interface Foo : NSObject <MouseListener> 
    {...}
  @end
  \end{lstlisting}
\end{frame}

\begin{frame}[fragile]
  Ainda é possível definir métodos opcionais num \textit{protocol}:
        \lstset{language=C++,basicstyle=\scriptsize}
  \begin{lstlisting}
  @protocol Slave
  
  @required
    -(void) makeCoffee;
    -(void) duplicateDocument:(Document*)document;
  
  @optional
    -(void) sweep;

  \end{lstlisting}
\end{frame}

\begin{frame}
  O \textit{protocol}, quando usado junto com objetos distribuidos, apresenta algumas palavras reservadas novas:
  \begin{itemize}
    \item[\textit{in}] - Parâmetro de input.
    \item[\textit{out}] - Parâmetro de output.
    \item[\textit{inout}] - Parâmetro de input e output.
    \item[\textit{bycopy}] - Parâmetro é copiado.
    \item[\textit{byref}] - Parâmetro é referênciado.
    \item[\textit{oneway}] - O método é assincrono.
  \end{itemize}
\end{frame}

\begin{frame}
  Ainda temos as \textit{categories} no Objective-C. \\
  Elas são utilizadas como \textit{protocols} informais. O programador pode agrupar métodos com a mesma finalidade numa mesma categoria e adicionar essa categoria
  numa classe. A utilidade principal é criar uma documentação para o programa. As categorias são ignoradas pelo compilador.
\end{frame}

\begin{frame}[fragile]
  Uma última resalva sobre \textit{protocols}, \textit{categories} e herança em Objective-C. \\
  Não é possível herdar uma classe e definir uma categoria na mesma linha, como por exemplo:
  \lstset{language=C++,basicstyle=\scriptsize}
  \begin{lstlisting}
  @interface Foo1 : SuperClass <Protocol1, Protocol2, ... > //ok
  @end
  
  @interface Foo2 (Category) <Protocol1, Protocol2, ... > //ok
  @end
  
  //below : compilation error
  @interface Foo3 (Category) : SuperClass <Protocol1, Protocol2, ... >
  @end
  
  //a solution :
  @interface Foo3 : SuperClass <Protocol1, Protocol2, ... > //1
  @end
  
  @interface Foo3 (Category) //2
  @end
  \end{lstlisting}
\end{frame}

%-------------------------------------
\subsection{Alocação}
\section{Alocação e Inicialização}
%-------------------------------------

\subsection{Alocação}
\begin{frame}
  \begin{columns}
    \begin{column}{.5\textwidth}
        \begin{center}
            C++
        \end{center}
        Objetos podem ser alocados na pilha como uma variável local, ou dinamicamente com o operador \textit{new}.
    \end{column}
    \begin{column}{.5\textwidth}
        \begin{center}
            Objective-C
        \end{center}
        Todos os objetos são alocados dinamicamente, enviando a mensagem \textit{alloc} para a classe.
    \end{column}
  \end{columns}
\end{frame}

\subsection{Inicialização}
\begin{frame}
  \begin{columns}
    \begin{column}{.5\textwidth}
        \begin{center}
            C++
        \end{center}
        Alocação e inicialização é um processo único, realizado sempre sequência. 
        É responsabilidade do \textit{construtor} inicializar o objeto. \\
        É possível ter mais de um construtor.
    \end{column}
    \begin{column}{.5\textwidth}
        \begin{center}
            Objective-C
        \end{center}
        A inicialização é realizada chamando um método, típicamente chamado de \textit{init}. \\
        É possível e usual ter outras métodos cujo nome apenas começa com init, e são alternativas de inicialização.
    \end{column}
  \end{columns}
\end{frame}

\begin{frame}[fragile]
  \begin{columns}
    \begin{column}{.5\textwidth}
      \begin{center}
        C++
      \end{center}
      \lstset{language=C++,basicstyle=\tiny}
      \begin{lstlisting}
      class Point2D {
        Point2D();
        Point2D(float x, float y);
      };
      
      Point2D p1(5, 4);
      Point2D* p2 = new Point2D;
      Point2D* p3 = new Point2D(5, 6);
      \end{lstlisting}
    \end{column}
    \begin{column}{.5\textwidth}
      \begin{center}
        Objective-C
      \end{center}
      \lstset{language=C++,basicstyle=\tiny}
      \begin{lstlisting}
        @interface Point2D
        // Metodos
        -(Point2D*) init;
        -(Point2D*) initWithX: (float) x andY: (float) y;
        @end
        
        Point2D* p1 = [[Point2D alloc] init];
        Point2D* p2 = [[Point2D alloc] 
                        initWithX: 5 andY: 6];
      \end{lstlisting}
    \end{column}
  \end{columns}
\end{frame}

\subsection{Memória}
\begin{frame}
  \begin{columns}
    \begin{column}{.55\textwidth}
        \begin{center}
            C++
        \end{center}
        Toda vez que um objeto é liberado, o seu \textit{destrutor} é chamado. \\
        Para objetos na pilha, eles são liberados automaticamente quando saem de escopo. \\
        Para objetos dinâmicos, é necessário usar o operator \textit{delete}.
    \end{column}
    \begin{column}{.55\textwidth}
        \begin{center}
            Objective-C
        \end{center}

        Objective-C utiliza contagem de referência, manipulado através dos métodos \textit{retain} 
        (incrementa em 1) e \textit{release} (decrementa em 1). \\
        Assim que o contador chega em 0, o destrutor \textit{dealloc} é chamado.
    \end{column}
  \end{columns}
\end{frame}

\end{document}
