\documentclass[brazil]{beamer}
\usepackage{beamerthemesplit}
\usepackage[brazilian]{babel}
\usepackage[utf8]{inputenc}
\usepackage{hyperref}
\usepackage{color}
\usepackage{xcolor}
\usepackage{amssymb}
\usepackage{amsmath}
\usepackage{fancybox}
\usepackage{ulem}
\usepackage{listings}
\usetheme{JuanLesPins}

\title{Objective-C vs C++}
\author{Thiago de Gouveia Nunes\\ Henrique Gemignani Passos Lima}

\begin{document}

\frame{\titlepage}

%-------------------------------------
\section{0. História!}
%-------------------------------------

\frame{
  \begin{center}
    \Huge Um pouco de história...
  \end{center}
}

\frame{
  Objective-C e C++ são "forks" da linguagem C com a implementação de POO, criadas em 1983.
  \begin{columns}
    \begin{column}{.5\textwidth}
      \begin{center}
        \textbf{C++} 
      \end{center}
      Implementação com código mais estático de POO, com foco em desempenho.
    \end{column}
    \begin{column}{.5\textwidth}
      \begin{center}
        \textbf{Objective-C}
      \end{center}
      Muito próxima do smalltalk-80 tanto em sintaxe quanto em dinâmismo.
    \end{column}
  \end{columns}
}

%-------------------------------------
\section{1. Semelhanças}
%-------------------------------------

\frame{
  \begin{center}
    \Huge Semelhanças
  \end{center}
}

\frame{
  \begin{itemize}
    \item Extensões de C: código em C pode ser compilado como Objective-C ou C++
    \item Suporte a programação orientada a objetos.
    \item Declaração e código podem ser misturados.
    \item Separação da declaração dos headers em 2 arquivos.
    \item Atributos \textit{public}, \textit{private}, \textit{protected}.
    \item Referênciação de memória.
  \end{itemize}
}

%-------------------------------------
\subsection{Fundamentos}
%-------------------------------------


\end{document}
